\documentclass{exam}
\usepackage[utf8]{inputenc}
\usepackage{amsmath}
\usepackage{amssymb}
\usepackage{graphics}
\usepackage{graphicx}
\title{Number theory Problems}
\author{CS/MATH 113 team}

\printanswers 
\begin{document}

\maketitle

\begin{questions}
    \question Prove that for all natural numbers $n>1$, $ \sqrt[n]{n}$ is irrational
    \begin{solution}
        Suppose $\sqrt[n]{n}$ is rational for some $n \in \mathbb{N}$
        \\Then there exists integers $a$ and $b$, such that $\sqrt[n]{n} = \frac{a}{b}$, where $b \neq 0$ and $gcd(a,b) =1$
        $$\sqrt[n]{n} = \frac{a}{b} \Rightarrow n = \frac{a^n}{b^n}$$
        $$gcd(a,b) =1 \Rightarrow gcd(a^n,b^n) =1$$
        As $n \in \mathbb{N}$, then $b^n = 1$, which means $n = a^n$
        \\As $n > 0$ and $b^n = 1$, then $a^n > 0$, which means that $a > 0$
        \\$a \neq 1$, as if $a = 1$ then $n = \frac{a^n}{b^n} = \frac{1}{1} = 1$, but $n > 1$, so $a \geq 2$
        \\We know for all natural numbers $n$ $2^n > n$ (this result is trivial and can be easily proved by mathematical induction.
        \\So $a^n \geq 2^n > n$, which means $n \neq a^n$, there we have a contradiction with out orignal claim that $n = a^n$
        \\Therefore for all natural numbers $n>1$, $ \sqrt[n]{n}$ is irrational
        \begin{flushright}
            $\square$
        \end{flushright}
    \end{solution}

    \question Given that $p$ is a prime and $p|a^n$, prove that $p^n|a^n$.
    \begin{solution}
        As $p|a^n$ then $a^n = kp$ for some integer $k$.
        \\\textbf{Case 1:} $p \neq a$
        \\Then $a$ is not a prime, then $a = p_1\times p_2 \times ... p_m$
        \\$a^n = p_1^n\times p_2^n \times ... p_m^n = kp$
        \\As $p|a^n$ and $a^n = p_1^n\times p_2^n \times ... p_m^n$ then there must be some $p_i$ from $1\leq i \leq m$ such that $p|p_i$
        \\As $p_i$ is prime for all $i \leq i \leq m$, then if $p|p_i$ then $p_i = p$ which means $p|a$
        \\Then $a = pq$ so $a^n = p^n q^n$ therefore $p^n|a^n$.
        \\\textbf{Case 2:} $p = a$ 
        \\If $p = a$  and $p|a^n$ then as $a^n|a^n$ and $a^n=p^n$ then $p^n|a^n$.
        \begin{flushright}
            $\square$
        \end{flushright}
    \end{solution}
    \pagebreak

    \question Show that any composite three-digit number must have a prime factor less than or equal to 31.
    \begin{solution}
        The next prime after 31 is 37, then the smallest composite number not containing a prime factor less than or equal to 31 would be $37^2 = 1369$ which is 4 digits.
        \begin{flushright}
            $\square$
        \end{flushright}
    \end{solution}

    \question Show that $\sqrt{p}$ is irrational for any prime number $p$.
    \begin{solution}
        Suppose $\sqrt{p}$ is rational then $\sqrt{p} = \frac{r}{q}$ where $q \neq 0$ and $gcd(q,r) = 1$
        \\Then $p = \frac{r^2}{q^2}$, so $pq^2 = r^2$
        \\Now as $r^2 = r \times r$ then any number in prime factorization of $r^2$ would appear an even number of times. 
        \\Similiary any number in prime factorization on $q^2$ appear and even number of times.
        \\So take $q^2 = p_1 \times p_2 \times... p_n \times p_1 \times p_2 \times... p_n$
        \\As $p|r^2$ and $q^2|r^2$ then $r^2 = p \times p_1 \times p_2 \times... p_n \times p_1 \times p_2 \times... p_n$
        \\Now $p$ is a number that appears in prime factorization of $r^2$ an odd number of times.
        \\Here we have a contradiction, therefore $\sqrt{p}$ is irrational.
        \begin{flushright}
            $\square$
        \end{flushright}
    \end{solution}
    
    \question Show that if $a$ is a positive integer and $\sqrt[n]{a}$ is rational, then $\sqrt[n]{a}$ must be an integer.
    \begin{solution}
        Let $a \in \mathbb{Z}^+$, suppose $\sqrt[n]{a}$ is rational, we show that then $\sqrt[n]{a}$ must be an interger.
        \\Let $\sqrt[n]{a} = \frac{p}{q}$, where $p,q \in \mathbb{Z}$ where $q \neq 0$ and $\text{gcd}(p,q) = 1$.
        $$\sqrt[n]{a} = \frac{p}{q} \Leftrightarrow a = \frac{p^n}{q^n} \Leftrightarrow a q^n = p^n$$
        Now we have that $q^n | p^n$, but as $\text{gcd}(p,q) = 1$ then $\text{gcd}(p^n,q^n) = 1$.
        \\So as only common divider of $p^n$ and $q^n$ is 1 and $q^n | p^n$ then $q^n = 1$
        \\Therefore $ a = \frac{p^n}{q^n} = p^n$, so $\sqrt[n]{a} = p$.
        \\Which means $sqrt[n]{a}$ is an integer.
        \begin{flushright}
            $\square$
        \end{flushright}
    \end{solution}
    
    \question In this question we will prove Euclid's Lemma that if $p$ is a prime number that divides $ab$ then $p$ divides $a$ or $p$ divides $b$.
    \\We shall prove this by proving a lemma and using a corollary from that lemma.
    \\\textbf{Well ordering principle:} Every non empty set of positive integers have a smallest element.
    \\\textbf{Division algorithm:} if $a,b \in \mathbb{Z}$, where $b>0$, then there exists unique $q,r \in \mathbb{Z}$, $a=bq+r$ where, $0 \leq r <b$ 
    \begin{parts}
    \part \textbf{Bezout's lemma:} for all integers $a$ and $b$ there exist integers $s$ and $t$ such that $gcd(a,b) = as + bt$
    \begin{solution}

        Let $S =\{am + bn \mid m,n\in \mathbb{Z} \mbox{ and } am+bn>0\}$
        \\Due to well ordering principle $S$ has a smallest element $d$
        $$d = as + bt$$
        We claim that $d=gcd(a,b)$
        \\Using the division algorithm $a = dq+r$, where $0\leq r<d$
        \\We assume $r>0$, and reach a contradiction, from which we can conclude that $r=0$ thus $d$ would divide $a$
        \\If $r>0$
        $$r = a - dq = a - (as + bt)q = a - asq - btq = a(1-sq)+b(-tq) \in S$$
        $r$ is in the form that it belongs to our set $S$, but as said above $r<d$ thus it contradicts the fact that $d$ is the smallest element in $S$
        \\Thus $r=0$, which means $d$ divides $a$
        \\Same argument can be constructed for $b$ and used to show that $d$ divides $b$ as well.
        \\Now assume there exist $d'$ that is also a divisor of $a$ and $b$.
        \\Let $a=d'h$ and $b = d'k$
        \\Then $d = as+bt= (d'h)s+(d'k)t=d'(sh +kt)$, then $d'$ is also a divisor of $d$
        \\Thus $d>d'$, so by universal generalization we can conclude that $d$ is the greatest of all divisors of $a$ and $b$.
        Thus contradiction with the fact that $d$ is the smallest element.
        \begin{flushright}
            $\square$
        \end{flushright}
    \end{solution}

    \part \textbf{Corollary of bezout's lemma:} If $a$ and $b$ are relatively prime then $as+bt = 1$

    \part Using the above corollary prove Euclid's lemma.
        \begin{solution}
            Let $p$ be a prime that divides $ab$ but does not divide $a$
            \\We need to show that $p$ must divide $b$
            \\As $p \nmid a$ and $p$ is a prime then $\text{gcd}(a,p) = 1$
            \\Then there exist $s,t\in \mathbb{Z}$ such that $1=as +pt$
            $$b=abs +pbt$$
            as $p$ divides right hand side then $p$ would divide $b$ as well.
            \begin{flushright}
                $\square$
            \end{flushright}
        \end{solution}
    \end{parts}

    \pagebreak
    \question For all positive integers $a$ and $b$ show that $\text{gcd}(a,b) \text{lcm}(a,b)=ab$.
    \begin{solution}
        Let $d = \text{gcd}$ for $a,b \in \mathbb{Z}$. Then $\exists p,q \in \mathbb{Z}$ s.t. $a = pd$ and $b = qd$.
        \\Let $m = \frac{ab}{d}$ then $m = aq = pb$. Which means $a|m$ and $b|m$ which mean $m$ is a common multiple of $a$ and $b$.
        \\Now we need to show that $m$ is indeed the least common multiple of $a$ and $b$.
        \\Let $c$ be a common multiple of $a$ and $b$, then $c = at = sb$.
        \\From bezout's lemma we know that $\exists x,y \in \mathbb{Z}$ s.t. $d = ax + by$.
        \\We show that $m|c$ which would imply that $m \leq c$.
        $$\frac{c}{m} =  \frac{cd}{ab} = \frac{c(ax + by)}{ab} = \frac{cax}{ab} + \frac{cby}{ab}$$
        $$\frac{cax}{ab} + \frac{cby}{ab} = \frac{cx}{b} + \frac{cy}{a} = \frac{c}{b}x + \frac{c}{a}y$$
        $$\frac{c}{m} = \frac{c}{b}x + \frac{c}{a}y = sx + ty$$
        As $s,x,t,y \in \mathbb{Z}$ then $sx + ty \in \mathbb{Z}$, which means $m|c$ theefore $m \leq c$.
        \\Which means $m$ is the least common multiple of $a$ and $b$.
        \\So we have that $dm = \text{gcd}(a,b) \text{lcm}(a,b) = ab$.
        \begin{flushright}
            $\square$
        \end{flushright}
    \end{solution}

    \question Show that there are infinitely many primes, in other words the set containing all prime numbers is infinite.
    \\\textbf{Definition:} A prime number is a Natural number that is only divisible by 1 and itself, and has to be divisible by 2 different numbers.
    \\\textbf{Fundamental Theorem of Arithmetic:} Every integer $N > 1$ has a prime factorization, meaning either $N$ is itself prime or can be written as a product of prime numbers.
    \begin{solution}
        Let $s=\{p_0,p_1,p_2,...,p_n\}$ be set of all primes. 
        \\Let $P = p_0 \times p_1 \times p_2 \times ... \times p_n$
        \\Let $q = P+1$
        \\\textbf{Case 1:}
        \\$q$ is prime, which is not in our set $s$
        \\\textbf{Case 2:} 
        \\if $q$ is not prime, then there exits a prime factor decomposition of $q$.
        \\Let $f$ be a prime that divides $q$, then $f$ would be in our set $s$ thus $f$ would divide $P$ too. 
        \\As $f$ divides $q$ and $P$ then $f$ divides $q-P$, which is $1$
        \\Then $f$ divides 1.
        \\As $f\geq2$ $f$ cannot divide 1, thus we have a contradiction.
        \begin{flushright}
            $\square$
        \end{flushright}
    \end{solution}

    \question Prove the following claim: There exists irrational numbers $a$ and $b$ such that $a^b$ is rational.
    \begin{solution}
        Take $a = \sqrt{2}$ and $b = \sqrt{2}$
        $$c = a^b$$
        \textbf{Case 1:}
        \\If $\sqrt{2} ^{\sqrt{2}}$ is rational then we already have our irrational numbers $a$ and $b$ such that $a^b$ is rational
        \\\textbf{Case 2:}
        \\If $\sqrt{2} ^{\sqrt{2}}$ is irrational then, let $a = \sqrt{2} ^{\sqrt{2}}$ and $b = \sqrt{2}$
        $$c = \left(\sqrt{2} ^{\sqrt{2}}\right)^{\sqrt{2}} = 2$$
        and 2 is rational
        \begin{flushright}
            $\square$
        \end{flushright}
    \end{solution}

    \question Show that $\sqrt{2}$ is irrational. In other words, $\sqrt{2}$ cannot be written in the form $\frac{p}{q}$ where $p,q \in \mathbb{Z}$ and $q \neq 0$
    \begin{solution}
        Assume $\sqrt{2}$ is rational, then $\sqrt{2} = \frac{p}{q}$, where $p,q \in \mathbb{Z}$ and $q \neq 0$.
        \\And $\frac{p}{q}$ is the lowest form it can be. 
        $$\left(\frac{p}{q}\right)^2 = 2$$
        $$p^2 = 2 q^2$$
        This implies $p$ is even which means $p = 2k$, for some $k \in \mathbb{Z}$
        $$4k^2 = 2 q^2$$
        $$2k^2 = q^2$$
        This implies $q$ is even.
        \\But $p$ and $q$ can't both be even as they are in the lowest form possible thus the 2 would be canceled. 
        \\Here we have a contradiction.
        \\Thus $\sqrt{2}$ cannot be written in form $\frac{p}{q}$ where $p,q \in \mathbb{Z}$
        \\Thus $\sqrt{2}$ is irrational.
        \begin{flushright}
            $\square$
        \end{flushright}
    \end{solution}

    

    \question Show that $x^n + y^n = z^n$ has no solutions where $x, y,z \in \mathbb{Z}$ with and $x \neq 0, \;\; y \neq 0,\;\;z \neq 0$ whenever $n\in \mathbb{Z}$ and $n > 2$
    \begin{solution}
        I've found a remarkable proof of this fact, but there is not enough space in the margin to write it. 
        \begin{flushright}
            $\square$
        \end{flushright}
    \end{solution}


    
\end{questions}

\end{document}