\documentclass{exam}
\usepackage[utf8]{inputenc}
\usepackage{amsmath}
\usepackage{amssymb}
\usepackage{graphics}
\usepackage{graphicx}
\title{Functions and Cardinality Problems}
\author{CS/MATH 113 team}

\printanswers 
\begin{document}

\maketitle

\begin{questions}
    \question Let $A$ be a set, show that $|A| < P(A)$, (remember that $A$ can be infinite).
    \begin{solution}
        
    \end{solution}

    \question Show that $F:\mathbb{Z} \rightarrow \mathbb{Z}$, where $f(x) = 2x+4$ is a bijection.
    \begin{solution}
        
    \end{solution}

    \question Show that $f(x) = |x|$ is not a bijection.
    \begin{solution}
        
    \end{solution}

    \question Show that bijection is an equivalence relation on all sets.
    \begin{solution}
        
    \end{solution}

    \question If $f : A \rightarrow B$ is a one-to-one function, show that there exists a function $g : B \rightarrow A$ such that $g$ is onto.
    \begin{solution}
        
    \end{solution}

    \question If $f : A \rightarrow B$ is a onto function, show that there exists a function $g : B \rightarrow A$ such that $g$ is one-to-one.
    \begin{solution}
        
    \end{solution}

    \question Show that for every one-to-one and onto function $f : A \rightarrow B$ it has an inverse function $g$ , such that $\forall a \in A,g(f(a)) = a$ and $\forall b \in B,f(g(b)) = b$.
    \begin{solution}
        
    \end{solution}

    \question Show that union of 2 countable sets is still countable.
    \begin{solution}
        
    \end{solution}

    \question Show that cartesian product of 2 countable sets is still countable.
    \begin{solution}
        
    \end{solution}

    \question Show that it is impossible to have one-to-one correspondence between a finite set and one of its proper subsets. 
    % (Hint: Suppose that it is possible to have 𝐶 ⊆ 𝐵, 𝐶 ≠ 𝐵, and a one-to-one function 𝑓: 𝐵 → 𝐶. Consider the set 𝐵 having the smallest number of elements for which this happens.)
    \begin{solution}
        
    \end{solution}

    \question Show that the natural numbers can be put in one-to-one correspondence with a proper subset of the natural numbers.
    \begin{solution}
        
    \end{solution}

    \question A prime number is a natural number that is divisible only 1 and itself.
    \\Let $\mathbb{P}$ be the set of all prime numbers. This set is infinite and a subset of $\mathbb{N}$, thus countably infinite. Let $P(\mathbb{P})$ be the power set of $\mathbb{P}$. Given that $P(\mathbb{P})$ is uncountable. Let $S$ is the set of all finite subsets of $\mathbb{P}$. Show that $P(\mathbb{P})$\textbackslash$S$ is uncountable.
    \begin{solution}
        
    \end{solution}
    
\end{questions}

\end{document}