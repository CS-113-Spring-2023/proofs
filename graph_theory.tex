\documentclass{exam}
\usepackage[utf8]{inputenc}
\usepackage{amsmath}
\usepackage{amssymb}
\usepackage{graphics}
\usepackage{graphicx}
\title{Graph theory Problems}
\author{CS/MATH 113 team}

\printanswers 
\begin{document}

\maketitle

\begin{questions}
    \question Let G be a Graph of Grith 4 in which every vertex has a degree k. 
    prove that G will have atleast 2k vertices. 
    \begin{solution}
        Let $G$ be a k-regular Graph, s.t. $grith(G) = 4$ and degree of each vertex in G is $k$. 
        \\Let $x$, $y$ be $2$ vertices in $G$ s.t. $x$ and $y$ are connected. 
        \\As grith of $G$ is 4, the length of the smallest cycle in $G$ is 4.
        \\Therefore $x$ and $y$ cannot have any common neighbors as that would create a cycle of 3. 
        \\The neighborhood of $x$ and $y$ are disjoint sets of size $k$, $N(x) = N(y) = k$
        \\Therefore there are atleaste $2k$ vertices in $G$
        \begin{flushright}
            $\square$
        \end{flushright}
    \end{solution}

    \question Number of vertices of odd degree in a graph is always even.
    \begin{solution}
        Let $G = (V,E)$ be a graph
        $$|E| = \frac{1}{2}\sum\limits_{v\in V} degree(v)$$
        Then $\sum\limits_{v\in V} degree(v)$ must be even, so if an odd number of vertices have odd degree then $\sum\limits_{v\in V} degree(v)$ would be odd.
        \\Therefore number of vertices of odd degree in a graph must be even.
        \begin{flushright}
            $\square$
        \end{flushright}
    \end{solution}
    
    \question Prove that for any graph $G$ and $H$, $G \cong H$ iff $\overline{G} \cong \overline{H}$.
    \begin{solution}
        Let $G = (V(G), E(G))$ and $H = (V(H), E(H))$ be 2 graphs.
        \\First we show that $G \cong H \implies \overline{G} \cong \overline{H}$
        \\Let $f:V(G) \rightarrow V(H)$ be an isomorphism from $G$ to $H$.
        \\As $f$ is an isomorphism $\forall u,v \in V(G)$, $uv\in E(G) \Leftrightarrow f(u)f(v)\in E(H)$.
        \\So for $\overline{G}$ we know $\forall u,v \in V(G)$, $uv\in E(\overline{G}) \Leftrightarrow uv \not\in E(G)$ 
        \\Then as for $uv\in E(G) \Leftrightarrow f(u)f(v)\in E(H)$, then under $f$, $uv\not\in E(G) \Leftrightarrow f(u)f(v)\not\in E(H)$
        \\So we have that $uv\in E(\overline{G}) \Leftrightarrow f(u)f(v)\not\in E(\overline{H})$
        \\So therefore $f$ is an isomorphism from $\overline{G}$ to $\overline{H}$.
        \\The same argumentcan be used to show the converse.
        \begin{flushright}
            $\square$
        \end{flushright}
    \end{solution}

    \question  Show that isomorphism of simple graphs is an equivalence relation.
    \begin{solution}
        We show that isomorphism is reflexive, symmetric and transitive.
        \\\textbf{Reflexive:} Let $G = (V,E)$, then we have bijection $f:V \rightarrow V$, where $\forall v \in V, \; v = f(v)$.
        \\Therefore $G \cong H$ 
        \\\textbf{Symmetric:}  Let $G = (V(G), E(G))$ and $H = (V(H), E(H))$,
        \\Let $f:V(G) \rightarrow V(H)$ be an isomorphism from $G$ to $H$, then inverse $f^{-1}:V(H) \rightarrow V(G)$ is an isomorphism from $H$ to $G$
        \\Therefore if $G \cong H$ then $H \cong G$
        \\\textbf{Transitive:} Let $G = (V(G), E(G))$, $H = (V(H), E(H))$ and $K = (V(K), E(K))$,
        \\Let $f:V(G) \rightarrow V(H)$ be an isomorphism from $G$ to $H$, and $g:V(H) \rightarrow V(K)$ be an isomorphism from $H$ to $K$, then $f \circ g$ is an isomorphism from $G$ to $K$
        \\Therefore if $G \cong H$ and $H \cong K$ then $G \cong K$
        \\\\So isomorphism is an equivalence relation.
        \begin{flushright}
            $\square$
        \end{flushright}
    \end{solution}

    \question A graph with maximum degree at most $k$ is $(k+1)$-colorable
    \begin{solution}
        Proof. We use induction on the number of vertices in the graph, which we denote by $n$. 
        \\\textbf{Base case:} A 1-vertex graph has maximum degree 0 and is 1-colorable, so $P(1)$ is true. 
        \\\textbf{Inductive Hypothesis:} Assume that a $n$-vertex graph with maximum degree at most $k$ is $(k+1)$-colorable.
        \\\textbf{Induction Step:} Let $G$ be an $(n+1)$-vertex graph with maximum degree at most $k$. 
        \\Remove a vertex $v$, leaving an $n$-vertex graph $G^{\prime}$. 
        \\The maximum degree of $G^{\prime}$ is at most $k$, and so $G^{\prime}$ is $(k+1)$-colorable by our Inductive Hypothesis. 
        \\Now add back vertex $v$. We can assign $v$ a color different from all adjacent vertices, since $v$ has degree at most $k$ neighbors and $k+1$ colors are available. 
        \\Therefore, $G$ is $(k+1)$-colorable. 
        \begin{flushright}
            $\square$
        \end{flushright}
    \end{solution}

    \question Show that $K_n$ has a Hamiltonian circuit whenever $n\geq 3$
    \begin{solution}
        We can form a Hamilton circuit in $K_n$ beginning at any vertex. Such a circuit can be
        built by visiting vertices in any order we choose, as long as the path begins and ends at the same
        vertex and visits each other vertex exactly once. This is possible because there are edges in $K_n$
        between any two vertices.
        \begin{flushright}
            $\square$
        \end{flushright}
    \end{solution}

    \question Prove that in every set of 6 people there is a set if atleast 3 mutual aquantainces or 3 mutual strangers.
    \begin{solution}
        For any graph of 6 vertices if the size of independent set is less than 3 than then atleast 3 vertices in the connected set and vise versa.
        \begin{flushright}
            $\square$
        \end{flushright}
    \end{solution}

    \question What are the number of edges in $k_n$.
    \begin{solution}
        As every vertex in $k_n$ has a degree $n-1$ then so the number of edges in $k_n$ are:
        $$\frac{n(n-1)}{2}$$

        \begin{flushright}
            $\square$
        \end{flushright}
    \end{solution}

    \question Show that every tree $T$ has atleast $\Delta(T)-1$ leaves, where $\Delta(T)$ represented the maximum degree of a vertex in $T$.
    \begin{solution}
        Let $T = (V,E)$ be a tree. Let $v \in V$, s.t. $d(v) = \Delta(T)$. As $T$ is a tree we know that $T$ has $\Delta(T)-1$ children.
        Then fron $v$ there are $\Delta(T)-1$ subtrees where $v$ is the root of those subtrees. Those subtrees would have atleast 1 leaf each. 
        As leaves of subtree are also leaves of $T$, therefore $T$ has atleast $\Delta(T)-1$ leaves.
        \begin{flushright}
            $\square$
        \end{flushright}
    \end{solution}

    \question Prove or disprove that if every vertex of a graph $G$ has a degree 2 then $G$ is a cycle.
    \begin{solution}
        $G$ can be 2 disconnected cycles, hence not true.
        \begin{flushright}
            $\square$
        \end{flushright}
    \end{solution}

    % \question Prove that every n-vertex graph with atleast n edges contains a cycle.
    % \begin{solution}
    %     % TODO
    %     \begin{flushright}
    %         $\square$
    %     \end{flushright}
    % \end{solution}

    \question Prove that every graph $G$ contains a path of length atleast $\delta(G)$ where $\delta(G)$ represented the smallest degree of a vertex in $G$.
    \begin{solution}
        Let $G = (V,E)$ be a graph. Let $P$ be the maximal path of $G$. Let $v$ be an endpoint of $G$. Then all neighbors of $v$ must be in $P$.
        Therefore $length(P)\geq d(v)$. As $d(v) \geq \delta(G)$, therefore $length(P)\geq \delta(G)$.
        \begin{flushright}
            $\square$
        \end{flushright}
    \end{solution}

    \question Prove that every graph $G$ contains a cycle of length atleast $\delta(G)+1$, where $\delta(G)$ represented the smallest degree of a vertex in $G$.
    \begin{solution}
        Let $G = (V,E)$ be a graph. Let $P$ be the maximal path of $G$. Let $v$ be an endpoint of $G$. Then all neighbors of $v$ must be in $P$.
        Therefore $length(P)\geq d(v)$. As $d(v) \geq \delta(G)$, therefore $length(P)\geq \delta(G)$. Let $u$ be the vertex in $P$ with maximum distance from $v$.
        Let $^uP_v$ be a $u$ to $v$ path in $P$. As all other neighbors of $v$ have smaller distance from $v$ than $u$, $length(^uP_v) \geq d(v) \geq \delta(G)$.
        Let $e = (u,v)$, then $^uP_v$ along with $e$ make a cyncle of length atleast $\delta(G)+1$.
        \begin{flushright}
            $\square$
        \end{flushright}
    \end{solution}

    \question \begin{parts}
        \part Show that if $G$ is a finite graph such that every vertex of $G$ has a degree of 2 then $G$ contains a cycle.
        \begin{solution}
            Let $G = (V,E)$ be a graph. Let $P$ be a maximal path of $G$. Let $u$ be an endpoint of $P$.
            As $d(u) = 2$ and $P$ cannoted be extended further, it must have a neighbor $v$ in $P$ with an edge $e$ not in $P$.
            Let $^uP_v$ be a $u$ to $v$ path in $P$. $^uP_v$ along with $e$ makes a cycle in $G$.
            \begin{flushright}
                $\square$
            \end{flushright}
        \end{solution}
        \part Show that this isn't nessesarily true if $G$ is an infinite graph.
        \begin{solution}
            Let $G=(V,E)$, where $V = \mathbb{Z}$, and $E = \{(i,j)|i,j \in \mathbb{Z}\;s.t.\;\; |i-j|=1\}$.
            $G$ every vertex of $G$ has a degree of 2 but $G$ doesn't contain a cycle.
            \begin{flushright}
                $\square$
            \end{flushright}
        \end{solution}
        \part Construct an infinite graph $G$ such that every vertex of $G$ has a degree of 2 and $G$ contains a cycle.
        \begin{solution}
            Let $G=(V,E)$, where $V = \mathbb{Z}^+$, and 
            $$E = \bigcup_{k\in \mathbb{Z}^+} = \{(3k,3k-1),(3k-1,3k-2),(3k-2,3k)\}$$
            $G$ every vertex of $G$ has a degree of 2 and $G$ contains cycles.
            \begin{flushright}
                $\square$
            \end{flushright}
        \end{solution}
    \end{parts}

    \question Prove that a graph is bipartite if and only if its 2 colorable.
    \begin{solution}
        First we show that if a graph $G$ is bipartite then $G$ is 2 colorable.
        \\Let $G = (V,E)$ be a bipartite graph, with bipartite sets $X$ and $Y$, assign all elements of $X$ some color $R$ and all elements of $Y$ some color $B$.
        As elements of $X$ has no neighbor in $X$ and elements of $Y$ has no neighbor in $Y$, then this is a valid coloring therefore $G$ is 2 colorable.
        \\Now we show that if $G$ is 2 colorable then $G$ is bipartite.
        \\Let $G = (V,E)$ be a 2 colorable graph (with 2 colors $R$ and $B$). Let $X$ be a set of all vertices of color $R$ and $Y$ be a set of all vertices of color $B$.
        \\As $R$ and $B$ color the graph no vertex in $X$ has a neighbor in $X$ and no vertex in $Y$ has a neighbor in $Y$, and $X \cup Y = V$. Therefore $G$ is bipartite.
        \begin{flushright}
            $\square$
        \end{flushright} 
    \end{solution}

\end{questions}

\end{document}